\chapter*{\abstractname} % Korrekter Name für den Abstract in der jeweiligen Sprache
\addcontentsline{toc}{chapter}{\abstractname}
Diese Vorlage soll als Erleichterung für alle dienen die, vernünftigerweise, ihre Abschlussarbeit in \LaTeX \ verfassen wollen. Hilfestellungen zu zu \LaTeX \ findet ihr unter folgenden Links:

\begin{center}
The Not So Short Introduction to \LaTeX: \url{http://tobi.oetiker.ch/lshort/lshort.pdf} \\
Wikibooks: \url{http://en.wikibooks.org/wiki/LaTeX/}
\end{center}

Als Editor empfehle ich TexMaker, da er für alle Betriebssysteme verfügbar ist und einige sehr angenehme Funktionen bietet.
\begin{center}\url{http://www.xm1math.net/texmaker/}\end{center}

Um dieses Dokument mit TexMaker kompilieren zu können sind folgende Einstellungen in Optionen -> Texmaker nötig:
\begin{itemize}
\item Die Option \enquote{Bib(la)tex} auf die biber.exe setzen. Diese sollte unter Windows irgendwo im Installationsordner von MikTex sein.
\item Unter \enquote{Quick Build} die User Option nehmen und mit dem Wizard die Reihenfolge
	\begin{enumerate}
	\item PdfLaTeX
	\item BibTex
	\item PdfLaTeX
	\item PdfLaTeX
	\item Pdf Viewer	
	\end{enumerate}
\end{itemize}
Damit kann die Option Quick Build komfortabel genutzt werden.\\

\noindent Unter Windows (MikTeX) unbedingt das Paket \textbf{cm-super} installieren.\\
Dieses beinhaltet Outline-Fonts welche - wenn installiert - automatisch statt den schlecht skalierbaren Bitmap-Fonts verwendet werden.

% Kein Seitenumbruch zwischen deutschem und englischem abstract
\let\oldcleardoublepage\cleardoublepage
\renewcommand\cleardoublepage{}

\addcontentsline{toc}{chapter}{Abstract}
\chapter*{Abstract}
Abstract in english.

\let\cleardoublepage\oldcleardoublepage
\newpage
