\RequirePackage{pdfmanagement-testphase}
\DeclareDocumentMetadata{pdfversion=1.7,uncompress,lang=de-DE}
\documentclass[12pt,a4paper,twoside,open=right,%
bibliography=totoc,BCOR=10mm]{scrreprt} % Schriftgröße, Seitenformat, Zweiseitig für Seitenränder, Bindcorrection
\usepackage{ifpdf}
\ifpdf
  \input{glyphtounicode.tex}    %Part of modern distribution
  \input{glyphtounicode-cmr.tex}     %Additionnal glyph: You must grab it from pdfx package
  \pdfgentounicode=1
  \pdfinterwordspaceon
  \usepackage[a-2u,pdf17]{pdfx}
  \pdfomitcharset 1
\else  %Place here the settings for other compilator
\fi
%Encoding + cmap (to get proper UTF8 mapping)
%------------------------------------------------------
\usepackage{cmap}
\usepackage[utf8]{inputenc} % Richtiges anzeigen von Umlauten und quasi allen anderen Schriftzeichen
\usepackage[T1]{fontenc} % Wichtig für alles was mehr als ASCII verwendet
\usepackage{csquotes} % Schöne Anführungsstriche mit \enquote{Text}
\usepackage{amsmath} % Bessere und schönere mathematische Formeln
\usepackage{mathtools} % Noch schönerere mathematische Formeln
\usepackage{amstext} % \text{} Macro in mathematischen Formeln
\usepackage{amsfonts} % Erweiterte Zeichensätze für mathematische Formeln
\usepackage{amssymb} % Spezielle mathematische Symbole.
%Correct UTF8 mapping for ams fonts
\ifdefined\pdffontattr% \ifdefined is part of the e-TeX extension, which is part of any modern LaTeX compiler.
    \immediate\pdfobj stream file {umsa.cmap}
    {\usefont{U}{msa}{m}{n}\pdffontattr\font{/ToUnicode \the\pdflastobj\space 0 R}}
    \immediate\pdfobj stream file {umsb.cmap}
    {\usefont{U}{msb}{m}{n}\pdffontattr\font{/ToUnicode \the\pdflastobj\space 0 R}}
\fi
\usepackage{array} % Matrizen in mathematischen Formeln
\usepackage{textcomp} % Für textmu und textohm etc. um im Fließtext keine Mathematik 
\usepackage{textalpha} % Damit können griechische Zeichen direkt im Text verwendet werden (siehe zeichen.txt)
\usepackage{paralist} % Für compactitem und compactenum
\usepackage{xstring} % Für IF in Titelseite

\usepackage[version=3]{mhchem} % Für Chemische Formeln
\usepackage{braket} % Für das quantenmechanische Bra-Ket

\usepackage{geometry} % Seitenränder und Seiteneigenschaften setzen
%\usepackage[showframe]{geometry} % Anzeigen der Seitenränder, nützlich für debugging. http://ctan.org/pkg/geometry

\usepackage[bottom]{footmisc} % Zwingt Fußnoten an das Ende der Seite
\usepackage[pdftex]{hyperref} % Links richtig anzeigen. Sowohl innerhalb des Dokuments (Fußzeilen, Formeln), als auch ins Internet

\usepackage[ % Biblatex für die Zitate und Referenzen
	backend=biber,
	hyperref=true
		]{biblatex}

\usepackage{xkeyval} % Erlaubt "Variablen" zu definieren, wird für Titelseite gebraucht
\usepackage{graphicx} % Wichtig für das Einbinden von Grafiken
\usepackage{caption}
\usepackage{subcaption} % Einbinden von mehreren Grafiken in einer figure

\usepackage{dirtree} % Erlaubt das erstellen von Dateibäumen
% \dirtreecomment{Text} erstellt einen Kommentar zu dem Verzeichnis bzw. der Datei
\newcommand{\dirtreecomment}[1]{\dotfill{} \begin{minipage}[t]{0.5\textwidth}#1\end{minipage}}

\usepackage{fancyvrb} % Mehr Optionen für Verbatim
\usepackage{listings} % Zur Darstellung von Programmcode
\usepackage{pdflscape} % Querformat Seiten

\newcommand{\writeIn}[1]{\usepackage[#1]{babel}} % Definiert einen neuen Befehl um die Sprache des Dokuments zu setzen

\usepackage{colorprofiles}
\PassOptionsToPackage{usenames,dvipsnames}{color}
\usepackage{color} % Farben für den todo Befehl
\newcommand{\todo}[1]{{\color{Cerulean}(TODO: #1)}} % Einfach \todo{Text} verwenden!

\newcommand{\blankpage}{ \newpage \thispagestyle{empty} \mbox{} \newpage }

\hypersetup{ % Setzt einige Werte die in den Eigenschaften des PDF gespeichert sind.
	bookmarksnumbered,
	pdfdisplaydoctitle = true,
	colorlinks, % Für Druck auf "false" setzen!
	linkcolor={black}, % auch keine grellen Rahmen im View
	citecolor={black},
	urlcolor={black}
}
%%%%%%%%%%%%%%%
%Marking the toc entries
%around the whole entry so only structure:
\newcommand\tagscrtocentry[1]{\tagstructbegin{tag=TOCI}#1\tagstructend}

%leaf so structure and mc:
\newcommand\tagscrtocpagenumber[1]{%
 \tagstructbegin{tag=Reference}%
 \tagmcbegin{tag=Reference}%
 #1%
 \tagmcend
 \tagstructend}

\DeclareTOCStyleEntry[
   entryformat=\tagscrtocentry,
   pagenumberformat=\tagscrtocpagenumber]{tocline}{chapter}
\DeclareTOCStyleEntry[
   entryformat=\tagscrtocentry,
   pagenumberformat=\tagscrtocpagenumber]{tocline}{section}
\DeclareTOCStyleEntry[
   entryformat=\tagscrtocentry,
   pagenumberformat=\tagscrtocpagenumber]{tocline}{subsection}
\DeclareTOCStyleEntry[
   entryformat=\tagscrtocentry,
   pagenumberformat=\tagscrtocpagenumber]{tocline}{subsubsection}
\DeclareTOCStyleEntry[
   entryformat=\tagscrtocentry,
   pagenumberformat=\tagscrtocpagenumber]{tocline}{paragraph}



\renewcommand{\addtocentrydefault}[3]{%
 \Ifstr{#3}{}{}
   {%\
   \Ifstr{#2}{}
    {%
     \addcontentsline{toc}{#1}
      {%
       \protect\nonumberline
       \tagstructbegin{tag=P}%
       \tagmcbegin{tag=P}%
        #3%
       \tagmcend
       \tagstructend
      }%
    }%
    {%
    \addcontentsline{toc}{#1}{%
     \tagstructbegin{tag=Lbl}%
     \tagmcbegin{tag=Lbl}%
     \protect\numberline{#2}%
     \tagmcend\tagstructend
     \tagstructbegin{tag=P}%
     \tagmcbegin{tag=P}%
      #3%
     \tagmcend
     \tagstructend
     }%
    }%
   }}%

% the dots must be marked too
\makeatletter
\renewcommand*{\TOCLineLeaderFill}[1][.]{%
  \leaders\hbox{$\m@th
    \mkern \@dotsep mu\hbox{\tagmcbegin{artifact}#1\tagmcend}\mkern \@dotsep
    mu$}\hfill
}

%%%%%%%%%
% Sectioning commands
%%%%%%%%

\ExplSyntaxOn
\prop_new:N   \g_tag_section_level_prop
\prop_gput:Nnn \g_tag_section_level_prop {chapter}{H1}
\prop_gput:Nnn \g_tag_section_level_prop {section}{H2}
\prop_gput:Nnn \g_tag_section_level_prop {subsection}{H3}
\prop_gput:Nnn \g_tag_section_level_prop {subsubsection}{H4}
\prop_gput:Nnn \g_tag_section_level_prop {paragraph}{H5}

%new 0.6, as attributes are local we have to put \tagmcbegin everywhere.
\renewcommand{\chapterlinesformat}[3]
 {
  \@hangfrom
   {
    \tagstructbegin{tag=\prop_item:Nn\g_tag_section_level_prop{chapter}}
    \tl_if_empty:nF{#2}
     {
      \tagmcbegin    {tag=\prop_item:Nn\g_tag_section_level_prop{chapter}}
      #2
      \tagmcend
     }
   }
   {\tagmcbegin    {tag=\prop_item:Nn\g_tag_section_level_prop{chapter}}
    #3\tagmcend\tagstructend}%
 }

%unnumbered sections level give an empty mc, need to think about it.
\renewcommand{\sectionlinesformat}[4]
 {
  \@hangfrom
   {\hskip #2
    \tagstructbegin{tag=\prop_item:Nn\g_tag_section_level_prop{#1}}
    \tl_if_empty:nF{#3}
    {
     \tagmcbegin    {tag=\prop_item:Nn\g_tag_section_level_prop{#1}}
     #3
     \tagmcend
    }
   }
   {\tagmcbegin    {tag=\prop_item:Nn\g_tag_section_level_prop{#1}}
    #4
    \tagmcend\tagstructend}%
 }

\ExplSyntaxOff
\AfterTOCHead{\tagstructbegin{tag=TOC}}
\AfterStartingTOC{\tagstructend} %end TOC

\writeIn{german} % Siehe header.tex. Setzt Dokumentsprache und damit Sprache von "Abstract", "Inhaltsverzeichnis", Datumsangaben etc.
\input{titlepage.tex}

\addbibresource{bib/example.bib}

\begin{document}
\tagstructbegin{tag=Document}

\AbschlussarbeitTUWienPhysikTitlePage{
	titel={Investigation of mysterious invisible planes},
	institut=Atominstitut,
	prof={\citeauthor{PersonBadurek}}, % citeauthor nimmt Namen der Person aus der Bibliography in bib/example.bib
	autor=Wonderwoman,
	adresse={Wiedner Hauptstrasse 8-10, Turm C,\\ 1. Stock, Raum 123A/124A},
	typ={dipl},
	% Vordefinierte Typen sind: sem (Seminararbeit), bacc (Bachelorarbeit), proj (Projektarbeit), mast (Diplomarbeit), dipl (Diplomarbeit), diss (Dissertation)
	% Laut Auskunft des Dekants muss auch eine Masterarbeit den Titel "Diplomarbeit" tragen.
	% Bei allen anderen Typen werden die Texte direkt übernommen.
	schwarzweisslogo=false % Definiert, ob das Logo der TU in Schwarz-Weiß oder Farbe ist.
	}

\pagenumbering{gobble} % Keine Seitenzahl drucken
\titlePageTUWienPhysik

\chapter*{\abstractname} % Korrekter Name für den Abstract in der jeweiligen Sprache
\addcontentsline{toc}{chapter}{\abstractname}
Diese Vorlage soll als Erleichterung für alle dienen die, vernünftigerweise, ihre Abschlussarbeit in \LaTeX \ verfassen wollen. Hilfestellungen zu zu \LaTeX \ findet ihr unter folgenden Links:

\begin{center}
The Not So Short Introduction to \LaTeX: \url{http://tobi.oetiker.ch/lshort/lshort.pdf} \\
Wikibooks: \url{http://en.wikibooks.org/wiki/LaTeX/}
\end{center}

Als Editor empfehle ich TexMaker, da er für alle Betriebssysteme verfügbar ist und einige sehr angenehme Funktionen bietet.
\begin{center}\url{http://www.xm1math.net/texmaker/}\end{center}

Um dieses Dokument mit TexMaker kompilieren zu können sind folgende Einstellungen in Optionen -> Texmaker nötig:
\begin{itemize}
\item Die Option \enquote{Bib(la)tex} auf die biber.exe setzen. Diese sollte unter Windows irgendwo im Installationsordner von MikTex sein.
\item Unter \enquote{Quick Build} die User Option nehmen und mit dem Wizard die Reihenfolge
	\begin{enumerate}
	\item PdfLaTeX
	\item BibTex
	\item PdfLaTeX
	\item PdfLaTeX
	\item Pdf Viewer	
	\end{enumerate}
\end{itemize}
Damit kann die Option Quick Build komfortabel genutzt werden.\\
\section*{Bitmap-Fonts}
\noindent Unter Windows (MikTeX) unbedingt das Paket \textbf{cm-super} installieren.\\
Dieses beinhaltet Outline-Fonts welche - wenn installiert - automatisch statt den schlecht skalierbaren Bitmap-Fonts verwendet werden.
\section*{Grafiken}
\noindent Diagramme sollten nach Möglichkeit als Vektorgrafiken eingebunden werden. Auch die nachträgliche Beschriftung von Rasterimages sollte in Illustrator, LibreDraw, oder ähnlich gemacht werden.
% Kein Seitenumbruch zwischen deutschem und englischem abstract
\let\oldcleardoublepage\cleardoublepage
\renewcommand\cleardoublepage{}

\addcontentsline{toc}{chapter}{Abstract}
\chapter*{Abstract}
Abstract in english.

\let\cleardoublepage\oldcleardoublepage
\newpage

\pdfbookmark[0]{Inhaltsverzeichnis}{Contents} % Inhaltsverzeichnis als PDF-Bookmark aber nicht im TOC
\tableofcontents \newpage
\cleardoublepage % Macht, dass openright funktioniert.
% \chapter macht das automatisch, \tableofcontents und \printbibliography machen das nicht.
% Falls es Probleme gibt hilft auch der Befehl \blankpage (siehe header.tex)
\pagenumbering{arabic} \setcounter{page}{1}

\chapter{Tipps und Tricks}
\tagstructbegin{tag=P}
        \tagmcbegin{tag=P}
Hier möchte ich kurz zusammenfassen, was ich alles beim Schreiben meiner Diplomarbeit gelernt habe. Das ist sicher weder eine vollständige Liste, noch eine die absolut richtig und der einzige Weg ist. Dennoch gibt sie im allgemeinen einen guten Anhalts und Startpunkt vor. Das zweite Kapitel \ref{chap:real-life} dient nur als kleines technisches Beispiel für Zitate, einbinden von Grafiken etc.
        \tagmcend
\tagstructend
\section{Aufbau}
\tagstructbegin{tag=P}
        \tagmcbegin{tag=P}
Eine Arbeit sollte mindesten die folgenden Teile enthalten:\\
\textbf{Titelseite}\\
\textbf{Kurzzusammenfassung/Abstract} Sollte idealerweise auf Deutsch und Englisch um die Suche in Bibliotheken zu vereinfachen.\\
\textbf{Inhaltsverzeichnis}\\
\textbf{Einleitung} Sollte einen kurzen (1 bis 2 Seiten) Überblick über die Problemstellung, das Experiment und den aktuellen Stand der Forschung geben.\\
\textbf{Hauptteil} Sollte den Großteil der Arbeit ausmachen.\\
\textbf{Conclusio und Ausblick} Sollte eine Zusammenfassung der Arbeit geben und darstellen was an neuem Wissen generiert wurde.\\
\textbf{Quellenverzeichnis} Hier ist speziell darauf zu achten, dass alles korrekt zitiert wird und man keine Plagiate produziert.
        \tagmcend
\tagstructend
\section{Roter Faden}
\tagstructbegin{tag=P}
        \tagmcbegin{tag=P}
Bevor man wirklich zu schreiben beginnt empfiehlt es sich einen \textbf{Roten Faden} zu skizzieren. Eine wissenschaftliche Arbeit sollte von der Leserin bzw. dem Leser keine Voraussetzungen erwarten und alles, über den Grundlagen liegende, Wissen erklären bzw. darauf referenzieren. Dementsprechend ist eine solche Arbeit mit einer Geschichte vergleichbar die von Beginn bis Ende schlüssig und zusammenhängend sein muss.
        \tagmcend
\tagstructend
\tagstructbegin{tag=P}
        \tagmcbegin{tag=P}
Um das zu erreichen lohnt es sich zuerst das Inhaltsverzeichnis auszuarbeiten und dieses mit der Betreuerin bzw. dem Betreuer abzusprechen. Danach sollte man zu allen Kapiteln und Unterkapiteln stichwortartig den Inhalt skizzieren. Ideal ist es wenn man auch hier die Abfolge miteinbezieht, z.B. Experimentaufbau -> Verwendete Geräte -> Funktion der Geräte -> Zusammenspiel erklären. Auf diese Weise lässt sich schon sehr früh erkennen wo die Fehler in der Struktur liegen und man kann noch leicht umordnen.
        \tagmcend
\tagstructend
\tagstructbegin{tag=P}
        \tagmcbegin{tag=P}
Jedenfalls sollte eine wissenschaftliche Arbeit keine chronologische Aufzählung sein. Viel eher sollte der Aufbau folgendermaßen sein:
\begin{compactitem}
\item Aufzeigen der Problemstellung
\item Methode zur Lösung des Problems präsentieren
\item Aufwertung und Darstellung der Daten
\item Interpretation der Ergebnisse
\item Diskussion, Vergleich mit alten Ergebnissen, Verbesserungsvorschläge
\end{compactitem}
        \tagmcend
\tagstructend

\chapter{Das wahre Leben}
\label{chap:real-life}
\tagstructbegin{tag=P}
        \tagmcbegin{tag=P}
\begin{center}
Is this the real life? \\
Is this just fantasy? \\
Caught in a landslide \\
No escape from reality \\
Open your eyes \\
Look up to the skies and see \\
I'm just a poor boy, I need no sympathy \\
Because I'm \enquote{easy come, easy go} \\
Little high, little low \\
Any way the wind blows, \\
doesn't really matter to me, 
to me \footnote{Auszug aus \citetitle{BohemRhap}~\cite{BohemRhap} von \citeauthor{Queen}~\cite{Queen} }\\
\end{center}
        \tagmcend
\tagstructend
\section{Farrokh Bulsara aka. Freddie Mercury}
\tagstructbegin{tag=P}
        \tagmcbegin{tag=P}
Farrokh Bulsara war ein Ausnahmetalent schuf zusammen mit der Band Queen einige der größten Hits aller Zeiten. Noch heute ist er ein wichtiges Thema in unterschiedlichsten Medien, wie in Abb. \ref{fig:freddiehg} zu sehen ist. Weitere Zitate sind in Anhang \ref{appendix:zitate} zu finden.
        \tagmcend
\tagstructend
\begin{figure}
\begin{center} \includegraphics{figures/freddie_mercury_by_wirdoudesigns.jpg} \end{center}
\caption{Freddie Mercury comic by~\citeauthor{wirdou}~\cite{wirdou}}
\label{fig:freddiehg}
\end{figure}

\newpage

\chapter{Mathematik}
\tagstructbegin{tag=P}
        \tagmcbegin{tag=P}
\bigskip
Issue with mapsto : ${\mathcal F} : \boldsymbol{\eta} \in {\mathbb{R}}^{np}\ \mapsto {\mathcal F}\left(\boldsymbol{\eta} \right)\in \mathbb{R}$

\bigskip
Issue with sqrt : $\sqrt{X}$

\bigskip
Issue with parenthesis : $X \geqslant \left(\frac{1}{2}\right)^2$

\bigskip
Issue with sum : $\sum_{n=0}^\infty X^n$
        \tagmcend
\tagstructend
\newpage


\cleardoublepage
\pagenumbering{roman} \setcounter{page}{1}
\printbibliography
\chapter*{Danksagung}
\tagstructbegin{tag=P}
        \tagmcbegin{tag=P}
Ein großer Dank für dieses Werk geht an die Community des Versionskontrollsystems GIT und dessen Erfinder Linus Torvalds. Ohne GIT wäre die Entwicklung, selbst einer so einfachen \LaTeX Vorlage sehr viel aufwändiger gewesen. Ich möchte auf diesem Weg auch allen die mit dieser Vorlage arbeiten empfehlen GIT zu verwenden. Es wird euer Leben sehr viel einfacher machen und ist noch dazu gratis und quelloffen inklusive Dokumentation und Buch verfügbar.
\begin{center}
GIT: \url{http://git-scm.com/} \\
Buch: \url{http://git-scm.com/book}
        \tagmcend
\tagstructend
\end{center}
\newpage


\appendix
\chapter{Zitate}
\label{appendix:zitate}
\tagstructbegin{tag=P}
        \tagmcbegin{tag=P}
We’re gonna stay together until we fucking well die, I’m sure of it. I keep—I must tell you—I keep wanting to leave, but they won’t let me. We’re not bad for four aging queens, really, are we? What do you think? \\
\enquote{Queen: Live at Wembley} (1986), shortly before performing \enquote{Who Wants To Live Forever.}
        \tagmcend
\tagstructend


\tagstructend   %Document
\end{document}