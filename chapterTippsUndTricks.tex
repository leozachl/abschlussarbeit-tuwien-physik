\chapter{Tipps und Tricks}
\tagstructbegin{tag=P}
        \tagmcbegin{tag=P}
Hier möchte ich kurz zusammenfassen, was ich alles beim Schreiben meiner Diplomarbeit gelernt habe. Das ist sicher weder eine vollständige Liste, noch eine die absolut richtig und der einzige Weg ist. Dennoch gibt sie im allgemeinen einen guten Anhalts und Startpunkt vor. Das zweite Kapitel \ref{chap:real-life} dient nur als kleines technisches Beispiel für Zitate, einbinden von Grafiken etc.
        \tagmcend
\tagstructend
\section{Aufbau}
\tagstructbegin{tag=P}
        \tagmcbegin{tag=P}
Eine Arbeit sollte mindesten die folgenden Teile enthalten:\\
\textbf{Titelseite}\\
\textbf{Kurzzusammenfassung/Abstract} Sollte idealerweise auf Deutsch und Englisch um die Suche in Bibliotheken zu vereinfachen.\\
\textbf{Inhaltsverzeichnis}\\
\textbf{Einleitung} Sollte einen kurzen (1 bis 2 Seiten) Überblick über die Problemstellung, das Experiment und den aktuellen Stand der Forschung geben.\\
\textbf{Hauptteil} Sollte den Großteil der Arbeit ausmachen.\\
\textbf{Conclusio und Ausblick} Sollte eine Zusammenfassung der Arbeit geben und darstellen was an neuem Wissen generiert wurde.\\
\textbf{Quellenverzeichnis} Hier ist speziell darauf zu achten, dass alles korrekt zitiert wird und man keine Plagiate produziert.
        \tagmcend
\tagstructend
\section{Roter Faden}
\tagstructbegin{tag=P}
        \tagmcbegin{tag=P}
Bevor man wirklich zu schreiben beginnt empfiehlt es sich einen \textbf{Roten Faden} zu skizzieren. Eine wissenschaftliche Arbeit sollte von der Leserin bzw. dem Leser keine Voraussetzungen erwarten und alles, über den Grundlagen liegende, Wissen erklären bzw. darauf referenzieren. Dementsprechend ist eine solche Arbeit mit einer Geschichte vergleichbar die von Beginn bis Ende schlüssig und zusammenhängend sein muss.
        \tagmcend
\tagstructend
\tagstructbegin{tag=P}
        \tagmcbegin{tag=P}
Um das zu erreichen lohnt es sich zuerst das Inhaltsverzeichnis auszuarbeiten und dieses mit der Betreuerin bzw. dem Betreuer abzusprechen. Danach sollte man zu allen Kapiteln und Unterkapiteln stichwortartig den Inhalt skizzieren. Ideal ist es wenn man auch hier die Abfolge miteinbezieht, z.B. Experimentaufbau -> Verwendete Geräte -> Funktion der Geräte -> Zusammenspiel erklären. Auf diese Weise lässt sich schon sehr früh erkennen wo die Fehler in der Struktur liegen und man kann noch leicht umordnen.
        \tagmcend
\tagstructend
\tagstructbegin{tag=P}
        \tagmcbegin{tag=P}
Jedenfalls sollte eine wissenschaftliche Arbeit keine chronologische Aufzählung sein. Viel eher sollte der Aufbau folgendermaßen sein:
\begin{compactitem}
\item Aufzeigen der Problemstellung
\item Methode zur Lösung des Problems präsentieren
\item Aufwertung und Darstellung der Daten
\item Interpretation der Ergebnisse
\item Diskussion, Vergleich mit alten Ergebnissen, Verbesserungsvorschläge
\end{compactitem}
        \tagmcend
\tagstructend
